\subsection{Architecture and control}

\newcommand{\ct}[1]{\texttt{#1}}

\subsubsection{OROCOS RTT-Lua}

The Orocos tool chain is a suite of tools allowing developers to
create complex real-time robotics applications using software
components. The Orocos real-time toolkit (RTT) is a C++ programming
framework included in the tool chain. Among other features RTT
provides a home-made scripting language and state machine
implementation to let developers implement components without
recompiling. Recently, the RTT developers have replaced their
home-made scripting language by a new one based on
Lua~\cite{Klotzbucher:2010fk}. According to the RTT developers Lua
provides a set of features that makes it a good scripting language in
the context of real-time robotics systems: indeed, Lua provides
advanced control on the memory allocation and garbage collection. Lua
is also seen as a good vehicle to write embedded DSLs as it is
dynamically typed, can execute an expression embedded within a string,
and provide a generic compound type.

Lua enabled the RTT developers to design a more powerful state machine
implementation as an embedded DSL within Lua. Implementing this DSL
requires using expressive language features of Lua which results in
memory allocation at run-time. As this is undesirable in a tight
real-time environment RTT developers present strategies to avoid this
issue. One of these strategies is to use a real-time allocator
associated with an estimation of the total amount of memory required.
The second strategy is to pre-allocate a block of memory reserved for
emergencies when the required memory exceed the estimation: During
such an emergency, the RTT framework uses this block of memory to stop
the robot in a safe manner and to make adjustments or perform manual
garbage collection.

RTT developers report on the following advantages of their approach
compared to the previous home-made scripting language and state
machine implementation~\cite{Klotzbucher:2010fk}. First, their
approach avoids premature optimizations by prototyping functionality
in a high-level language. Second, the use of Lua allows for
portability of component scripts and state machines into other
frameworks. Finally, the approach increases robustness as programming
errors in scripts are limited to the respective interpreter.

\subsubsection{URBI}

Each robot comes with its own programming language interface. From
this observation Baillie~\cite{Baillie:2005} developed a universal
robotic low-level programming language named URBI. This DSL does not
impose any philosophy or architecture on the developer. Instead, URBI
is an imperative language which emphasizes event programming and
parallelism.

In URBI, reading a sensor or controlling an actuator is respectively
done by reading a variable or writing into one. Where URBI
distinguishes itself is by its use of \emph{modificators} that affect
how a variable is written. The following instruction (taken
from~\cite{Baillie:2005}) commands a robot's head pan to reach the
value 80 degrees in 4500ms:

\begin{verbatim}
headPan.val = 80 time:4500;
\end{verbatim}

With the \ct{time} modificator, URBI progressively changes the value
of a variable from its current value to its target value within the
specified time. Other modificators allow to specify, for example, a
speed at which a particular value is changed or a sinusoidal
oscillation around a value with a particular period and amplitude.

Additionally URBI proposes four operators to control how a set of
instructions is executed (in strict sequence, in strict parallel, in
flexible sequence, and in flexible parallel). These operators can also
be attached to loops: for example \ct{for \&} starts all iterations of
the loop in parallel.

Executing multiple instructions in parallel can potentially result in
conflicts when two instructions change the same variable. URBI
proposes six strategies to control the result of conflicting
assignment. % For example, one such strategy is called \ct{mix} and
% allows to average the conflicting assignments. As a result, a developer
% can write the following code to average the values of an array:

% \begin{verbatim}
% blend(mix) m;
% for & (i=0;i<10;i++)
%   m = tab[i];
% \end{verbatim}
For example, the \ct{add} strategy can be used to generate a turning
behavior while a walk is performed or to play two sounds at the same
time.

According to Baillie, all these features of URBI allow developers to
write programs more quickly and more efficiently, and in a portable way.


\begin{itemize}
\item Component-based approach for robotics: 
  ROCK\footnote{\url{http://rock-robotics.org/}} \cite{Joyeux:2011fk}
\item FRP (Functional reactive
  programming)\footnote{\url{http://www.cs.jhu.edu/\~{}hager/Public/ICRAtutorial/HagerPeterson-FRP/ICRA02-tutorial.pdf}}:
  Functional Robotics (FROB)\cite{Hager:1999fk},
  \cite{Peterson:1999lk}, \cite{Pembeci:2002fc}
\item Misc \cite{Thiry:2008ys}, \cite{Proetzsch:2010vn}
\end{itemize}

% LocalWords:  Orocos RTT Lua DSL DSLs URBI modificator
