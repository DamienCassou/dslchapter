\subsection{Architecture and control}

The Orocos tool chain is a suite of tools allowing developers to
create complex real-time robotics applications using software
components. The Orocos real-time toolkit (RTT) is a C++ programming
framework included in the tool chain. Among other things RTT provides
a home-made scripting language and state machine implementation to let
developers implement components without recompiling. Recently, the RTT
developers have replaced their home-made scripting language by a new
one based on Lua. According to~\cite{Klotzbucher:2010fk} Lua provides
a set of features that makes it a good scripting language in the
context of real-time robotics systems: indeed, Lua provides advanced
control on the memory allocation and garbage collection. Lua is also
seen as a good vehicle to write embedded DSLs as it is dynamically
typed, can execute an expression embedded within a string, and provide
a generic compound type. Lua enabled the RTT developers to design a
more powerful state machine implementation as an embedded DSL within
Lua.

\begin{itemize}
\item Component-based approach for robotics: OROCOS
  RTT-LUA\cite{Klotzbucher:2010fk},
  ROCK\footnote{\url{http://rock-robotics.org/}} \cite{Joyeux:2011fk}
\item Orchestration languages for robotics:
  urbiscript\cite{Baillie:2005},
\item FRP (Functional reactive
  programming)\footnote{\url{http://www.cs.jhu.edu/\~{}hager/Public/ICRAtutorial/HagerPeterson-FRP/ICRA02-tutorial.pdf}}:
  Functional Robotics (FROB)\cite{Hager:1999fk},
  \cite{Peterson:1999lk}, \cite{Pembeci:2002fc}
\item Misc \cite{Thiry:2008ys}, \cite{Proetzsch:2010vn}
\end{itemize}

% LocalWords:  Orocos RTT Lua DSL DSLs
