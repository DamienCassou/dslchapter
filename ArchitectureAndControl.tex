\subsection{Architecture and control}

\subsubsection{OROCOS RTT-Lua}

The Orocos tool chain is a suite of tools allowing developers to
create complex real-time robotics applications using software
components. The Orocos real-time toolkit (RTT) is a C++ programming
framework included in the tool chain. Among other things RTT provides
a home-made scripting language and state machine implementation to let
developers implement components without recompiling. Recently, the RTT
developers have replaced their home-made scripting language by a new
one based on Lua~\cite{Klotzbucher:2010fk}. According to the RTT
developers Lua provides a set of features that makes it a good
scripting language in the context of real-time robotics systems:
indeed, Lua provides advanced control on the memory allocation and
garbage collection. Lua is also seen as a good vehicle to write
embedded DSLs as it is dynamically typed, can execute an expression
embedded within a string, and provide a generic compound type.

Lua enabled the RTT developers to design a more powerful state machine
implementation as an embedded DSL within Lua. Implementing this DSL
requires using expressive language features of Lua which results in
memory allocation at run-time. As this is undesirable in a tight
real-time environment, RTT developers present various strategies to
avoid this issue. One of these strategies is to use a real-time
allocator associated with an estimation of the total amount of memory
required. The second strategy is to pre-allocate a block of memory
reserved for emergencies when the required memory exceed the
estimation: In such an emergency, this block of memory is used to stop
the robot in a safe manner and to make adjustments or perform manual
garbage collection.

RTT developers report on the following advantages of their approach
compared to the previous home-made scripting language and state
machine implementation~\cite{Klotzbucher:2010fk}. First, their
approach avoids premature optimizations by prototyping functionality
in a high-level language. Second, the use of Lua allows for
portability of component scripts and state machines into other
frameworks. Finally, robustness is improved as programming errors in
scripts are limited to the respective interpreter.


\begin{itemize}
\item Component-based approach for robotics: OROCOS
  RTT-LUA\cite{Klotzbucher:2010fk},
  ROCK\footnote{\url{http://rock-robotics.org/}} \cite{Joyeux:2011fk}
\item Orchestration languages for robotics:
  urbiscript\cite{Baillie:2005},
\item FRP (Functional reactive
  programming)\footnote{\url{http://www.cs.jhu.edu/\~{}hager/Public/ICRAtutorial/HagerPeterson-FRP/ICRA02-tutorial.pdf}}:
  Functional Robotics (FROB)\cite{Hager:1999fk},
  \cite{Peterson:1999lk}, \cite{Pembeci:2002fc}
\item Misc \cite{Thiry:2008ys}, \cite{Proetzsch:2010vn}
\end{itemize}

% LocalWords:  Orocos RTT Lua DSL DSLs
