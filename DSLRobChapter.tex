\documentclass[doc]{apa}
%% More information about the APA style are available here: ftp://ftp.comp.hkbu.edu.hk/pub/TeX/CTAN/macros/latex/contrib/apa/apacls.html

\title{Trends in Domain-Specific Languages for Robotics System}
\fourauthors{Damien Cassou}{Christian Schlegel}{Ulrik P. Schultz}{Serge Stinckwich}
\fouraffiliations{HPI, University of Postdam, Germany}{Hochschule Ulm, Germany}{University of Southern Denmark, Denmark}{UMMISCO, UMI 209\\IRD/IFI/Vietnam National University}

\abstract{The abstract should be here.}

\acknowledgements{Bla bla ...}

\shorttitle{Trends in DSL for Robotics System}
\rightheader{Trends in DSL for Robotics System}
\leftheader{D.\ Cassou, C.\ Schlegel, U.\ Schultz, S.\ Stinckwich}

\begin{document}
\maketitle
One of the dreams of humanity since the ancient time is to be able to create machines that are skilled and intelligent.
This dream is becoming true with robotics\cite{SpringerHandbook:2008fk} which has made unquestionable progresses in the last decades.
Robotics are already making a considerable impact in everyday life from industrial manufacturing to health care, transportation, and exploration of unsafe environments for humans like deep space or nuclear plants.
Tomorrow, robots are envisioned to be as pervasive as today’s personal computer. The improvements in energy storage now allow a simple and light battery to power demanding sensors and actuators for hours. The availability of inexpensive and highly integrated embedded application processors, a consequence of the massive market of mobile phones and game devices like Microsoft’s Kinect, has brought advanced capabilities to mobile robots, such as high-resolution vision. Moreover, the improvements in electronics and hardware have lowered the cost of sensors and actuators, and have improved their quality and reliability. The combination of all these advances has thus led to affordable robot hardware and allowed sophisticated control software.

Despite all these advances, robotics systems are still difficult to design and deploy. This originates from the fact that robotics systems need to control diverse sensors and actuators in real time mode, in the face of significant uncertainty and noise. These systems must achieve their tasks while monitoring for, and reacting to, unanticipated situations. Doing all this concurrently and asynchronously adds immensely to system complexity. Adding to this complexity, robotics systems are often embedded in resource-constraint systems which are difficult to program because of lacking suitable tools and abstractions. Indeed, most of the available tools are quite low-level (C-like). Finally, all robotics systems contain orthogonal variabilities that further add to the overall complexity. For example, moving a robot can be done through either a differential or omnidirectional drive depending on the embedded hardware, and a robot’s temporal scope can range from a millisecond for feedback control to hours for mapping an environment.

This complexity makes robotics system software difficult to adapt and reuse in the context of different robots and scenarios. As a result, commercial robotics systems and research prototypes have either been limited to trivial tasks, such as obstacle detection, or have been restricted to certain kinds of tasks, hardware, or environment. Several software engineering approaches have been proposed to lower the complexity of robotics systems [2]. These approaches include robotics software frameworks (e.g., CLARATy robotic software framework\footnote{\url{http://claraty.jpl.nasa.gov/man/overview/index.php}}), robotics middleware systems (e.g., ROS\footnote{\url{http://www.ros.org/}}), component-based software engineering for robotics\cite{Brugali:2007oq} and model-driven software engineering for Robotics (e.g., OMG RTC\footnote{OMG RTC. Robotic Technology Component (RTC) Specification 1.0, 2008. \url{http://www.omg.org/spec/RTC}}, SmartSoft [10]). All these approaches apply and tailor general-purpose and established principles of lowering complexity to robotics needs and come up with domain-specific extensions.

Domain-Specific Languages (DSLs)\cite{Deursen:2000uq} and Model-Driven Architecture are emerging areas of interest in the robotics research community (see also the recent workshops on these topics [4], [7]). Both have been instrumental for resolving complex issues in a wide range of domains, including distributed and modular robotics, control, and vision. A DSL is a language dedicated to a particular problem domain offering specific notations and abstractions to increase developers productivity and collaboration. Models offer high-level ways for developers to specify the functionality of their system at the right level of abstraction. Robotics systems blend hardware and software in a way that raises many crosscutting concerns that general-purpose programming languages have traditionally had problems with. DSLs and models offer a powerful and systematic way to overcome this problem, enabling the programmer to quickly and precisely implement novel software solutions to complex problems within the robotics domain.

The main purpose of this chapter is to understand why domain-specific languages are interesting in the context of robotics. We will first identify existing DSLs in the robotics community (organized around the main concerns of robotics as listed in \cite{SpringerHandbook:2008fk}). We will then describe some features of DSLs that are particularly useful for the robotics domain such as run-time changes [6] and multiple-DSL integration [4]. We underpin these by robotics examples like URDF (descriptions of robot physical structures\footnote{\url{http://www.ros.org/wiki/urdf}}), task coordination [11], and model-driven software development for robotics (SmartSoft\footnote{\url{http://smart-robotics.sf.net/}}). Finally we will outline some research perspectives in this domain.

\section{Existing Robotics Domain-Specific Languages} % (fold)
\label{sec:Existing_Robotics_Domain-Specific_Languages}
We will first identify existing DSLs in the robotics community (organized around the main concerns of robotics as listed in \cite{SpringerHandbook:2008fk}).
\subsection{Robot Physical Structures and kinematics}
Part B "Robot Structures" of Springer Handbook of Robotics.

\subsubsection{Unified Robot Description Format (URDF)\footnote{\url{http://www.ros.org/wiki/urdf}}}
XML format to represent robot model (Kinematic and dynamic description, visual representation, collision model).
A robot description is usually composed by a set of links elements and a set of joint elements.
ROS contains an URDF parser that allows to build a C++ model from an XML file.
xacro is an XML macro-language suitable to build large URDF files.

\subsubsection{Paper\cite{Brugali:2007uq}}
\subsubsection{Paper\cite{Frigerio:2011fk}}

\subsection{Architecture and control}

\newcommand{\ct}[1]{\texttt{#1}}

\subsubsection{OROCOS RTT-Lua}

The Orocos tool chain is a suite of tools allowing developers to
create complex real-time robotics applications using software
components. The Orocos real-time toolkit (RTT) is a C++ programming
framework included in the tool chain. Among other features RTT
provides a home-made scripting language and state machine
implementation to let developers implement components without
recompiling. Recently, the RTT developers have replaced their
home-made scripting language by a new one based on
Lua~\cite{Klotzbucher:2010fk}. According to the RTT developers Lua
provides a set of features that makes it a good scripting language in
the context of real-time robotics systems: indeed, Lua provides
advanced control on the memory allocation and garbage collection. Lua
is also seen as a good vehicle to write embedded DSLs as it is
dynamically typed, can execute an expression embedded within a string,
and provide a generic compound type.

Lua enabled the RTT developers to design a more powerful state machine
implementation as an embedded DSL within Lua. Implementing this DSL
requires using expressive language features of Lua which results in
memory allocation at run-time. As this is undesirable in a tight
real-time environment RTT developers present strategies to avoid this
issue. One of these strategies is to use a real-time allocator
associated with an estimation of the total amount of memory required.
The second strategy is to pre-allocate a block of memory reserved for
emergencies when the required memory exceed the estimation: During
such an emergency, the RTT framework uses this block of memory to stop
the robot in a safe manner and to make adjustments or perform manual
garbage collection.

RTT developers report on the following advantages of their approach
compared to the previous home-made scripting language and state
machine implementation~\cite{Klotzbucher:2010fk}. First, their
approach avoids premature optimizations by prototyping functionality
in a high-level language. Second, the use of Lua allows for
portability of component scripts and state machines into other
frameworks. Finally, the approach increases robustness as programming
errors in scripts are limited to the respective interpreter.

\subsubsection{URBI}

Each robot comes with its own programming language interface. From
this observation Baillie~\cite{Baillie:2005} developed a universal
robotic low-level programming language named URBI. This DSL does not
impose any philosophy or architecture on the developer. Instead, URBI
is an imperative language which emphasizes event programming and
parallelism.

In URBI, reading a sensor or controlling an actuator is respectively
done by reading a variable or writing into one. Where URBI
distinguishes itself is by its use of \emph{modificators} that affect
how a variable is written. The following instruction (taken
from~\cite{Baillie:2005}) commands a robot's head pan to reach the
value 80 degrees in 4500ms:

\begin{verbatim}
headPan.val = 80 time:4500;
\end{verbatim}

With the \ct{time} modificator, URBI progressively changes the value
of a variable from its current value to its target value within the
specified time. Other modificators allow to specify, for example, a
speed at which a particular value is changed or a sinusoidal
oscillation around a value with a particular period and amplitude.

Additionally URBI proposes four operators to control how a set of
instructions is executed (in strict sequence, in strict parallel, in
flexible sequence, and in flexible parallel). These operators can also
be attached to loops: for example \ct{for \&} starts all iterations of
the loop in parallel.

Executing multiple instructions in parallel can potentially result in
conflicts when two instructions change the same variable. URBI
proposes six strategies to control the result of conflicting
assignment. % For example, one such strategy is called \ct{mix} and
% allows to average the conflicting assignments. As a result, a developer
% can write the following code to average the values of an array:

% \begin{verbatim}
% blend(mix) m;
% for & (i=0;i<10;i++)
%   m = tab[i];
% \end{verbatim}
For example, the \ct{add} strategy can be used to generate a turning
behavior while a walk is performed or to play two sounds at the same
time.

According to Baillie, all these features of URBI allow developers to
write programs more quickly and more efficiently, and in a portable way.

\subsubsection{Frob}

Frob is a domain-specific language, embedded in Haskell, for
functional robotics\cite{Hager:1999fk}. Frob provides a framework for
rapid prototyping and software reusability. To do so, the framework
includes two essential abstractions, \emph{behaviors} and
\emph{events}, and allow developers to build new abstractions on top
of these two. These two abstractions form the basis for a functional
reactive programming style. A behavior is an abstraction for a
function over continuous time. For example, a behavior can define a
wall-following algorithm. The syntax used to define behaviors is very
similar to the syntax used to define mathematical equations which
makes behaviors concise and manipulable by domain experts. An event is
used to control the execution of a behavior. For example, the
reception of a timeout event can stop a wall-following algorithm. From
these two abstractions developers are invited to create new ones that
more closely relate to the robotics applications they want to develop.
As an example, Hager and Peterson~\cite{Peterson:1999lk} define the
\emph{task} abstraction: a task is a behavior that terminates on an
event and returns a value. Frob also proposes functions, called
transformations, that creates a new task from an existing one by
combining it with a new event or behavior. In addition, the authors
show how new architectural styles, such as the subsumption
architecture, can be created on top of Frob. With these abstractions
and Haskell features to manipulate functions, Hager and Peterson
believe that Frob makes developing, testing, and executing robotics
application easy.



\begin{itemize}
\item Component-based approach for robotics: 
  ROCK\footnote{\url{http://rock-robotics.org/}} \cite{Joyeux:2011fk}
\item FRP (Functional reactive
  programming)\footnote{\url{http://www.cs.jhu.edu/\~{}hager/Public/ICRAtutorial/HagerPeterson-FRP/ICRA02-tutorial.pdf}}:
  Functional Robotics (FROB)\cite{Hager:1999fk},
  \cite{Peterson:1999lk}, \cite{Pembeci:2002fc}
\item Misc \cite{Thiry:2008ys}, \cite{Proetzsch:2010vn}
\end{itemize}

% LocalWords:  Orocos RTT Lua DSL DSLs URBI modificator Hager Frob
%  LocalWords:  subsumption

\subsection{Sensing and perception}
Uncertainty, probabilistic robotics: Lambda0, \cite{Thrun:2000}
\subsection{Visualisation and simulation}
Player-Stage configuration files\cite{Collett:2005bq}
\subsection{Distributed and networked robotics}
\begin{itemize}
\item MDL (Motion Description Language): \cite{Martin:2008vn}, \cite{Martin:2010uq}
\item PRONTO (spatial computing): \cite{Bachrach:2008bs}
\item cooperative robotics: \cite{Welborn:2005dg}
\item reconfigurable robotics: \cite{Schultz:2007le}, \cite{Schultz:2008vl}
\end{itemize}
\subsection{Planning}
\begin{itemize}
\item PDDL (Planning Domain Definition Language)
\item Smach\footnote{\url{http://www.ros.org/wiki/smach}} (finite state machines)
\end{itemize}
% section Existing_Robotics_Domain-Specific_Languages (end)

\bibliography{DSLRobChapter}

\end{document}